\documentclass[12pt, letterpaper]{article}
\usepackage[utf8]{inputenc}
\usepackage{amsmath}
\usepackage{amssymb}
\usepackage{amsfonts}
\usepackage{listings}
\usepackage{xcolor}
\usepackage{dirtree}
\usepackage{hyperref}
\usepackage{setspace}
\hypersetup{
    colorlinks,
    citecolor=black,
    filecolor=black,
    linkcolor=black,
    urlcolor=black
}

\newcommand*{\QEDB}{\null\nobreak\hfill\ensuremath{\square}}%
\newcommand\tab[1][1cm]{\hspace*{#1}}
\newcommand\tabd[1][0.2cm]{\hspace*{#1}}

\lstset{
  language     = C++,
  basicstyle   = \small,
  keywordstyle = \color{black}\textbf,
  commentstyle = \color{blue},
  stringstyle  = \color{green!70!black},
  stringstyle  = \color{gray},
  columns      = fullflexible,
  numbers      = left,
  numberstyle  = \scriptsize\sffamily\color{gray},
  showstringspaces = false,
  float,
}

\author{
        Joao Felipe Bianchi Curcio\\
        Jonas Edward Tashiro\\ 
        Luan Lopes Barbosa de Almeida\\ 
        Rafael Melloni Chacon Arnone\\ 
       }

\date{}
\title{
    \Huge{Calculo de Integrais}
}

\begin{document}

\maketitle
%\begin{center}
%\hypertarget{loretta}{}
%\end{center}

\newpage
\tableofcontents
\newpage

\section{Organização de Pastas}

\dirtree{%
  .1 src.
  .2 exp.
  .3 {definitions.h}.
  .3 {expression.h}.
  .3 {lexicon.h}.
  .3 {token.h}.
  .3 {lexicon.cpp}.
  .3 {expression.cpp}.
  .3 {token.cpp}.
  .2 integral.
  .3 {integral.h}.
  .3 {integral.cpp}.
  .2 {Makefile}.
  .2 {main.cpp}.
}

\section{Conteudo dos Arquivos}
\subsection{Pasta src}

\begin{spacing}{2.5}
\end{spacing}

\begin{lstlisting}[caption=main.cpp]
#include "exp/expression.h"
#include "exp/token.h"
#include "integral/integral.h"
#include <ios>
#include <ostream>
#include <string>
#include <iostream>
#include <utility>
#include <vector>
#include <limits>
#include <iomanip>

int main() {
    bool quit = false;
    bool cannot_be_conv = false;
    std::string user_expr;
    std::string user_info1, user_info2;
    int numTrapz, precision;
    double start, end;
    std::vector<std::pair<double, double>> fnTable;

    Expression *expression;
    do
    {
        std::cout << "\nEscreva uma expressao:\n";
        std::cout << "f(x) = ";
        std::getline(std::cin, user_expr);
        user_expr += ';';
        std::endl(std::cout);

        if(!addSpaces(user_expr))
        {
            expression = new Expression(user_expr);
            if(expression->isValid())
            {
                expression->infixToPostfix();
                do{
                    std::cout << "\nEscreva o ponto extremo da esquerda:\n";
                    std::cin >> user_info1; 
                    std::cout << "Escreva o ponto extremo da direita:\n";
                    std::cin >> user_info2; 
                    try{
                        start = std::stold(user_info1.data()); 
                        end = std::stold(user_info2.data()); 
                        cannot_be_conv = false;
                        user_info1.clear(), user_info2.clear();
                    }catch (std::invalid_argument){
                        std::cout << "Uma variavel foi escrita de 
                                      forma indevida";
                        cannot_be_conv = true;                    
                    }
                }while (cannot_be_conv);

                do {
                    std::cout << "\nEscreva o numero de casas decimais\n"; 
                    std::cin >> user_info1; 
                    std::cout << "Escreva o numero de trapezios:\n";
                    std::cin >> user_info2; 
                    try {
                        precision = std::stoi(user_info1.data());
                        numTrapz = std::stoi(user_info2.data());
                        cannot_be_conv = false;                    
                        user_info1.clear(), user_info2.clear();
                    } catch (std::invalid_argument) {
                        std::cout << "Uma variavel foi escrita de 
                                      forma indevida";
                        cannot_be_conv = true;
                    }
                }while (cannot_be_conv);

                std::cout << std::fixed;
                std::cout << std::setprecision(precision + 1);
                
                TrapezoidIntegral integralCalc(start,end,numTrapz,precision);
                integralCalc.calculateIntegral(*expression, fnTable);

                std::cout << "\n\nSaida:\n\n";
                std::cout << "Erro de arredondamento = " 
                          << integralCalc.errorRounding;

                std::cout << std::fixed;
                std::cout << std::setprecision(precision);

                std::cout << "\nValor da Integral = " 
                          << integralCalc.sumTraps;
                std::cout << "\nTabela de Valores:\n";
                std::cout << 'x'; 

                for (int i = -1; i <= precision; i++)
                    std::cout << ' ';

                std::cout << "| " << "f(x)" << '\n';
                for (auto f : fnTable)
                    std::cout << f.first << " | " << f.second << '\n';
                delete expression;
            }
            else
                std::cout << "\nA expressao fornecida contem erro\n";
        }
        else
            std::cout << "\nErro Lexico detectado na expressao fornecida\n";

        std::cout << "Deseja sair do programa?\n";
        std::cout << "Sim (Escreva q)\t Nao (Escreva qualquer coisa)\n";
        std::cin >> user_info1;
        if(user_info1 == "q")
            quit = true;

        std::cin.ignore(std::numeric_limits<std::streamsize>::max(),'\n');
        user_expr.clear();
        user_info1.clear(); 
        std::cout << "\033[2J\033[1;1H";

    }while (!quit); 
    return 0;
}
\end{lstlisting}

\newpage

\tabd Este proximo arquivo tem como intuito facilitar o processo de compilação.
\begin{lstlisting}[caption=Makefile]
EXPRESSION = ./exp
INTEGRAL = ./integral
SOURCES = main.cpp
SOURCES += $(EXPRESSION)/expression.cpp $(EXPRESSION)/lexicon.cpp 
SOURCES += $(EXPRESSION)/token.cpp
SOURCES += $(INTEGRAL)/integral.cpp
OBJECTS = $(addsuffix .o, $(basename $(notdir $(SOURCES))))


COMPILE : LINK 
	g++ -o main $(OBJECTS) 

LINK:
	g++ -c $(SOURCES)

clean:
	rm $(OBJECTS)
  
\end{lstlisting}

\newpage
\subsection{Pasta exp}

\begin{spacing}{2.5}
\end{spacing}

\begin{lstlisting}[caption=definition.h]
#pragma once
#include <utility>
#define FAILURE false
#define SUCCESS true

using AtributeValue = int;
using Priority = int;
using Token_name = int;
using Token_type = 
    enum : int
    {
        endExpression,
        leftParen,
        rightParen,
        unaryOp,
        binaryOp,
        operand,
        number
    };

using Token = std::pair<Token_name,AtributeValue>;
\end{lstlisting}

\begin{spacing}{3.5}
\end{spacing}

\begin{lstlisting}[caption=expression.h]
#pragma once
#include <queue>
#include <list>
#include <map>
#include <string>
#include <utility>
#include "definitions.h"
#include "lexicon.h"

using ErrorCode = bool;

class Expression
{
    std::list<Token> tokenized_expr;
    Lexicon symbol_table;
    public:
        Expression();
        Expression(std::string &expression); //Tokenize the expression
        ErrorCode infixToPostfix(); //certify that expression is valid first
        ErrorCode evaluateAt(double x, double &f_of_x);
        void tokenizeExpression(std::string &expression);
        void getIteratorRange(std::list<Token>::iterator &start,
                              std::list<Token>::iterator &end); 
        ErrorCode isValid();     //check for infix
    private:
        void removeFirstToken(); //move back to private
        void addToken(Token &new_token);
        float do_unary(double x, Token_name type);
        float do_binary(double x, double y, Token_name type);
};

/*Autenticate lexical correctness of expression before sending to the class*/
/*Also add whitespace to better identify lexemes*/
ErrorCode addSpaces(std::string &expression);

\end{lstlisting}

\begin{spacing}{3.5}
\end{spacing}

\begin{lstlisting}[caption=token.h]
#pragma once
#include "definitions.h"
#include <string>

struct Token_data
{
    std::string name;
    double value;
    Priority priority;
    Token_data(std::string token_name, double value, Priority priority);
};
\end{lstlisting}

\begin{spacing}{3.5}
\end{spacing}

\begin{lstlisting}[caption=lexicon.h]
#include <map>
#include <string>
#include <vector>
#include "definitions.h"
#include "token.h"
#define NON_EXISTENT -1

class Lexicon
{
    //shall only be used to setup expression
    std::map<std::string, AtributeValue> lexeme_map; 
    std::vector<Token_data*> symbol_table;
    public:
        Lexicon() = default;
        void setStandardTokens();   //sets up the map to lexemes
        Token_data* getTokenInfo(AtributeValue token_id); 
        AtributeValue newToken(Token_data *new_token);
        AtributeValue findAtribute(std::string &lexeme);
};
\end{lstlisting}

\newpage
\subsection{Pasta integral}

\begin{spacing}{2.5}
\end{spacing}

\begin{lstlisting}[caption=integral.h]
#include "../exp/expression.h"
#include <utility>
#include <vector>

struct TrapezoidIntegral
{
    int nOfTrapz;
    int precision;
    double errorRounding;
    double sumTraps;
    double x_start, x_end;
    void calculateIntegral(Expression &expr,
        std::vector<std::pair<double, double>> &fnTable);
    TrapezoidIntegral() = default;
    TrapezoidIntegral(double start, double end, int num, int precision);
};
\end{lstlisting}

\begin{spacing}{3.5}
\end{spacing}

\begin{lstlisting}[caption=integral.cpp]
#include "integral.h"
#include <cmath>
#include <math.h>
#include <utility>
#include <vector>

void setNumber(int precision, double &value);

TrapezoidIntegral::TrapezoidIntegral(double start, double end,
                                  int num, int precision)
{
    x_start = start;
    x_end = end;
    nOfTrapz = num;
    this->precision = precision;
}

void TrapezoidIntegral::calculateIntegral(Expression &expr,
        std::vector<std::pair<double, double>> &fnTable)
{
    double increment = std::abs(x_end - x_start);
    increment /= static_cast<double>(nOfTrapz);
    double x = x_start;
    double f_late, f_early;
    fnTable.clear();
    sumTraps = 0.0F;

    expr.evaluateAt(x, f_late);
    setNumber(precision,f_late);
    fnTable.push_back(std::pair<double, double>(x, f_late));
    x += increment;
    expr.evaluateAt(x, f_early);
    setNumber(precision, f_early);

    for (int i = 1; i <= nOfTrapz; i++)
    {
        fnTable.push_back(std::pair<double, double>(x, f_early));
        x += increment;
        sumTraps += increment * (f_early + f_late) / 2.0F;
        f_late = f_early;
        expr.evaluateAt(x, f_early);
        setNumber(precision, f_early);
    }    

    errorRounding = nOfTrapz * 
                  (5.0F / std::pow(10.0F, precision + 1)) * increment;
}

void setNumber(int precision, double &value)
{
    value *= pow(10.0, precision);
    value = std::round(value);
    value /= pow(10.0, precision);
}
\end{lstlisting}

\end{document}
